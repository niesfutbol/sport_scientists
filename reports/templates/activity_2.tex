\documentclass{article}
\usepackage[utf8]{inputenc} 
\usepackage[spanish,es-tabla,es-nodecimaldot]{babel}
\usepackage{graphicx}
\usepackage{amsmath}
\usepackage{csvsimple}
\usepackage{authblk}
\usepackage{cprotect}
\usepackage{floatrow}
\usepackage[caption=false]{subfig}
\usepackage{booktabs}
\usepackage{hyperref}
\usepackage{lscape}
\usepackage{gensymb}
\usepackage[a4paper,top=3cm,bottom=2cm,left=3cm,right=3cm,marginparwidth=1.75cm]{geometry}
\usepackage{siunitx}
\usepackage[square,sort,comma,numbers]{natbib}
\usepackage[nottoc,numbib]{tocbibind}
\usepackage{float}
\usepackage{pgfplotstable}
\floatstyle{plaintop}
\restylefloat{table}



\author{Nepo Rojas}

\title{Introducción a la Programación en R para Análisis de Datos Físicos \\ \begin{large}
    Actividad 2: Visualización de test de Salto Contramovimiento \end{large}}

\begin{document}

\maketitle

El código con el que generamos los resultados de este reporte lo encontrarás
\href{https://github.com/niesfutbol/sport_scientists/blob/develop/src/activity_2.R}{sport\_scientists/src/activity\_2.R}.
El código podría diferir del presente en este reporte. La versión actualizada del código está en la
liga al repositorio.

\subsubsection*{Relación entre las variables Potencia media concéntrica y la Profundidad del contramovimiento}
La figura \ref{fig:step2} nos permite explorar la relación entre las variables
\texttt{concentric\_mean\_power\_bm} y \texttt{countermovement\_depth}.
\begin{figure}[H]
\centering
\includegraphics[scale=0.6]{../results/step2.png}
\caption{Relación entre las variables \textbf{Potencia media concéntrica} y la \textbf{Profundidad
del contramovimiento}.}
\label{fig:step2}
\end{figure}

\subsubsection*{Serie de tiempo del RSI modificado}
La figura \ref{fig:step3} nos permite ver la tendencia de los valores de \texttt{rsi\_modified}. Los
los datos con los que generamos la figura corresponden al jugador 4317.
\begin{figure}[H]
\centering
\includegraphics[scale=0.6]{../results/step3.png}
\caption{Serie de tiempo del RSI modificado para el jugador 4317.}
\label{fig:step3}
\end{figure}
\subsubsection*{El pico de la fuerza concéntrica}
La figura \ref{fig:step4} hace un acomodo de los valores para la variable \texttt{concentric\_peak\_force}
de mayor a menor de los jugadores. El Jugador 2993 está señalado en un color distinto. Los datos
son del día 2021-02-26.
\begin{figure}[H]
\centering
\includegraphics[scale=0.6]{../results/step4.png}
\caption{El pico de la fuerza concéntrica de los jugadores para el día 2021-02-26. El jugador 2993 está resaltado en otro color.}
\label{fig:step4}
\end{figure}

La figura \ref{fig:step5} representa la misma información a la figura \ref{fig:step4} pero los datos
de los jugadores están agrupados por posición.

\begin{figure}[H]
\centering
\includegraphics[scale=0.6]{../results/step5.png}
\caption{El pico de la fuerza concéntrica para el día 2021-02-26 para los jugadores de cada posición. El jugador 2993 está resaltado en otro color.}
\label{fig:step5}
\end{figure}

\end{document}
