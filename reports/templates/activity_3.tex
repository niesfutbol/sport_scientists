\documentclass{article}
\usepackage[utf8]{inputenc} 
\usepackage[spanish,es-tabla,es-nodecimaldot]{babel}
\usepackage{graphicx}
\usepackage{amsmath}
\usepackage{csvsimple}
\usepackage{authblk}
\usepackage{cprotect}
\usepackage{floatrow}
\usepackage[caption=false]{subfig}
\usepackage{booktabs}
\usepackage{hyperref}
\usepackage{lscape}
\usepackage{gensymb}
\usepackage[a4paper,top=3cm,bottom=2cm,left=3cm,right=3cm,marginparwidth=1.75cm]{geometry}
\usepackage{siunitx}
\usepackage[square,sort,comma,numbers]{natbib}
\usepackage[nottoc,numbib]{tocbibind}
\usepackage{float}
\usepackage{pgfplotstable}
\floatstyle{plaintop}
\restylefloat{table}



\author{Nepo Rojas}

\title{Introducción a la Programación en R para Análisis de Datos Físicos \\ \begin{large}
    Actividad 3: Aceleración por posición \end{large}}

\begin{document}

\maketitle

El código con el que generamos los resultados de este reporte lo encontrarás
\href{https://github.com/niesfutbol/sport_scientists/blob/develop/src/activity_3.R}{sport\_scientists/src/activity\_3.R}.
El código podría diferir del presente en este reporte. La versión actualizada del código está en la
liga al repositorio.

\begin{figure}[H]
    \centering
    \includegraphics[scale=0.6]{../static/activity_3_1.png}
    \caption{
        Valores del jugador Pro 12. En cuadro azul vemos los datos de GPS. En el cuadro verde vemos
        la posición del jugador Pro 12.
    }
    \label{fig:1}
    \end{figure}
En la figura \ref{fig:1} vemos algunos de los valores del jugador Pro 12.

\begin{figure}[H]
    \centering
    \includegraphics[scale=0.6]{../static/activity_3_2.png}
    \caption{
        Valor promedio de la velocidad ($\frac{m}{s}$) y de la acceleration density (valor
        promedio de la aceleración absoluta, $\frac{m}{s^2}$). En el cuadro verde vemos el código
        fuente. En el cuadro azul vemos los valores promedio por jugador.
    }
    \label{fig:2}
    \end{figure}
En la figura \ref{fig:2} vemos el promedio de la velocidad y la acelerción para los jugadores.
Podemos ver que el jugador Pro 14 es el que tiene la velocidad promedio menor.
Es el jugador Pro 16 el que tiene la aceleración promedio menor.
El jugador Pro 17 es el que tiene los promedios de aceleración y velocidad mayores.

\begin{figure}[H]
    \centering
    \includegraphics[scale=0.6]{../static/activity_3_3.png}
    \caption{
        Valor promedio de la velocidad ($\frac{m}{s}$) y de la acceleration density (valor
        promedio de la aceleración absoluta, $\frac{m}{s^2}$) por posición. En el cuadro azul vemos el código
        fuente. En el cuadro verde vemos los valores promedio por posición.
    }
    \label{fig:3}
    \end{figure}
En la figura \ref{fig:3} vemos el promedio de la velocidad y la acelerción para las posiciones.
Vemos que son los delanteros los que tienen los promedios en la velocidad y aceleración mayores.
También podemos ver que son los defensas los que tienen el promedio en la velocidad más baja.
Son los medios los que tienen el promedio más bajo en la aceleración.

\end{document}
